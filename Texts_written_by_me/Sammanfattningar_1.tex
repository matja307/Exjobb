% !TEX encoding = MacOSRoman

\documentclass[a4paper,12pt]{article}

\usepackage{listings}
\usepackage[parfill]{parskip}
\usepackage{float}
\usepackage{graphicx}
\usepackage[swedish]{babel}
\usepackage[applemac]{inputenc}
\usepackage[margin=35mm]{geometry}
\usepackage{fancyhdr}
\usepackage[encapsulated]{CJK}
\usepackage{inconsolata}
\usepackage{amssymb,amsmath}
%\usepackage[demo]{graphicx}
\usepackage{caption}
\usepackage{subcaption}
\usepackage{cite}



\newenvironment{ppl}{\fontfamily{ppl}\selectfont}{}
\newenvironment{inconsolata}{\texttt}{}


\DeclareGraphicsRule{.tif}{png}{.png}{`convert #1 `dirname #1`/`basename #1 .tif`.png}
\addto\captionsswedish{%
  \renewcommand{\figurename}{Figure}%
  \renewcommand{\tablename}{Table}%
    \renewcommand{\contentsname}{Table of contents}%
}

\begin{document}

\newpage
\pagenumbering{gobble}
\tableofcontents

 \newpage
\pagenumbering{arabic}
\section{Lateral Enlargement Growth Mechanism of 3C-SiC on Off-Oriented 4H-SiC Substrates}
\emph{Jokubavicius, Valdas
Yazdi, G Reza
Liljedahl, Rickard
Ivanov, Ivan G
Yakimova, Rositsa}

This paper discusses the growth mechanism of 3C SiC used at LiU. It describes the experimental details of the growth chamber and conditions. It further discusses how the growth occurs from a physical point of view. It also discusses how it was concluded experimentally that the grown material is of good quality. 

The article states that 4H and 6H polytypes are commercially available, and that these materials are grown using PVD. The polytype 3C cannot be fabricated in the same way, since there are no \emph{seeds} available for this [9]. (The reference [9] describes how 4H and 6H is fabricated using PVD, where to get large volumes of single crystal SiC one uses seeds, small pieces of single crystal SiC).

The material SiC in general is stated as a good candidate for medium power devices, solar cells, bio-compatible material, see refs in paper. 

3C can in principle be created on a Si layer [10], but when this is done the quality of the fabricated material is bad [11]. A better candidate for the fabrication of 3C is $\alpha$-SiC. This is because $\alpha$-SiC has several properties which are similar to 3C. These are lattice parameter, thermal expansion coefficient and chemical compatibility (what is this?). 

3C is in this article created on off-axis $\alpha$-SiC. The substrate is cut not exactly at a crystal face. This means that there will be several steps in the substrate. This is good because the cubic material is then grown along the direction of the steps, called the step-flow direction. The 3C nucleates at one side of the sample in small islands. These are then extended to meet each other and then the step-flow growth begins. As the nucleation sites meet, the crystal orientation may not match and we get defects called \emph{double position boundaries}[12]. 

At the beginning of the growth a large terrace of 4H is created on the substrate. It is on this terrace that the 3C is formed. 

Using PL one can deduce the quality of the grown material. We can also get information about stress from the phonon replica lines. 

\newpage
\section{Stability Of Bulk Cubic Silicon Carbide (3C-SiC) On
Off Oriented Hexagonal Silicon (4H-SiC) Substrate (Exjobb)}
\emph{Po-Hsun Chen}

There are several factors which influence the growth rate: Supersaturation level (what is this?), Si/C-ratio and type of substrate are some [8].

The growth rate of CVD is much slower than for fast sublimation growth process (FSGP). The rates are 10 $\mu$m/h and 1 mm/h respectively. The FSGP works by having a temperature gradient, where the substrate is cooler than the source. Since there are many carbon components in the setup, for example the crucible, we will get graphitisation of the sample unless we do something about this. To avoid this we use a Ta-foil in the setup, which will adsorb some carbon. The FSGP uses dynamic vacuum, which means that the pump is turned on during the growth process. 

The paper discusses temperature dependence on result. When increasing the temperature we get more step bunching. 

\newpage
\section{Application of LTPL Investigation Methods to CVD-Grown SiC}
\emph{Jean Camassel, Sandrine Juillaguet, Marin Zielinski, and Carole Balloud}

This article describes two ways of finding the concentration of donor atoms in SiC. The two ways originate from the carrier-rate equation for semiconductors, which shows that the change in number of carriers depends on the generation rate, the carrier life times for both free and bound excitons (See paper for details). 

\[\frac{\mathrm{d}n_{\mathrm{FE}}}{\mathrm{d}t}  = g_{\mathrm{FE}}-\left(\frac{1}{\tau^r_{\mathrm{FE}}}+\frac{1}{\tau^{nr}_{\mathrm{FE}}}\right)n_{\mathrm{FE}}-t_{\mathrm{RX}}\]

The fact that we look at excitons and not free EHP-transitions is because at very low temperatures the excitonic recombinations are the most intense. The first method uses the ratio of the emission intensities of free and donor-bound excitons. In the paper they show that there will be a linear dependence between the ratio and the donor concentration, [N]:

\[\frac{I_{\mathrm{DX}}}{I_{\mathrm{FE}}}=K\times [N]\]

This method works well for 4H and 6H samples, but less well for 3C. This is because the FE-line is rarely visible in 3C due to the long lifetime of free excitons. The second method is then better, where we use a standard curve of the FWHM value of the TA-line. 

\newpage
\section{Combined effects of Ga, N, and Al codoping in solution grown 3C�SiC}
\emph{Sun, J. W.
Zoulis, G.
Lorenzzi, J. C.
Jegenyes, N.
Peyre, H.
Juillaguet, S.
Souliere, V.
Milesi, F.
Ferro, G.
Camassel, J.}

This paper deals with the estimation of de acceptor concentration levels in 3C SiC, using the method outlined in Camassel et. al. 

The 3C crystals have a wurtzite (ZnS) structure. Nitrogen is a common donor material, which has a binding energy of 54 meV [12]. Common acceptors are B, Al, Ga, In. A few of these, particularly Al, can frequently be found in SiC as unintentional dopant. This is because graphite is commonly used in the setup when fabricating SiC, and graphite usually contains some Al. 

If we assume that we have two different types of acceptors, 1 and 2, then the ration between the concentrations can be found in a similar way as in Camassel et. al. 

\[\frac{I(DAP_1)}{I(DAP_2)} = \frac{[A_1]R^{\tau}(DAP_1)}{[A_2]R^{\tau}(DAP_2)}\]

Here the ratio between the radiative and non-radiative lifetimes of the excitons have been taken into account through the R$^{\tau}$. In the paper this ratio has been approximated to be approximately 10. However the paper then states that the above formula is not enough to properly find the concentration ratio. We need also to take into account the effective radius of the acceptor atoms when capturing a carrier. It is of course so that if the radius is large, then it will be easier for that acceptor to capture an exciton and this will thus increase the emission from that acceptor. The radius depends on the binding energy of that acceptor, which is found by Haynes rule. See paper for details. 

The paper concludes by stating that the method still does not account for everything in the process, but that the approximation is good. 

\newpage
\section{Considerably long carrier lifetimes in high- quality 3C-SiC ( 111 )}
\emph{Sun, Jianwu
Ivanov, Ivan Gueorguiev
Liljedahl, Rickard
Yakimova, Rositsa
Syv�j�rvi, Mikael}

The paper shows that the life time of carriers in the grown 3C is high, 8.2 $\mu$s. It further shows that the grown material is of very high quality and that there is very little strain in the material. 

The paper uses XRD and $\omega$ rocking curves to find the lattice constant. Comparing this constant to table values for 3C it is shown that the strain in the crystal is very low. One can also use the ratio of $I_{\mathrm{LA}}/I_{\mathrm{TO}}$ to estimate the stress. If this ratio is near unity, then the stress is small. 

The life times are estimated using the photoconductive decay method (PCD). The method used uses microwave pulses, $\mu$-PCD. Using the data from the PCD measurements, the effective life time of the carriers can be found. The carrier lifetime is generally lower near the surface of the sample, and near the interface of the substrate and the sample. The bulk lifetime is larger, giving a total effective lifetime somewhere in between the different values. 

It is not much discussed in the article, but if we can see the higher excitonic complexes in the PL spectrum, then this indicates a high lifetime and thus a pure sample. The excitonic complexes are usually denoted with $\alpha_i$ and the excitons with excited electrons are denoted $\beta_i$. Ref [28] discusses this in more detail. 

Higher injection rate of carriers usually means higher surface recombination and this shorter lifetime near the surface. This is because the band is bent near the surface, separating the electrons and holes, but if we increase the carrier injection rate the band is straightened. 

\newpage
\section{Wikipedia on water splitting}

Water splitting is one way to realise a \emph{hydrogen economy}, where hydrogen is used as an energy source for powering motors or as local storage of energy. This energy can be stored in e.g. buildings as a source of energy for that particular building. The term hydrogen economy was coined by John Bockris in the 70s. 

In water splitting, water is split using electrolysis. This means that a current is passed through the water. This is explained in more detail in later papers, see below. 

\section{Photoelectrolysis of water to hydrogen in in p-SiC/Pt and p-SiC/n-TiO2 cells}
\emph{Akikusa, Jun
Khan, Shahed U M}

The band gap for the photoelectrode needs to be right. It needs to be large enough to both oxidice and reduce the water. 

A problem when using n-doped material as electrode is the photocorrosion, which means that the created holes are not used to oxidize the water but rather the SiC itself. 

How is this done? The SiC is oxidized in combination with the water, so that SiO and SiO$_2$ is formed? Or is it simply that the holes make the Si and C split?

\newpage
\section{Semiconductor-based Photocatalytic Hydrogen Generation}
\emph{Xiaobo Chen, Shaohua Shen, Liejin Guo, and Samuel S. Mao}

The concept of water splitting is not new. Using UV-light to split water has been done for some time. But since the UV part of the solar spectrum is only 4 \%, it is better if we can use the visible (VIS) part of the spectrum. A classical example from ref [18] is using TiO$_2$ and Pt electrodes. 

The process is that the incident light is absorbed and an ehp is created. This ehp is split due to a built in electric field in the electrode. (This electric field is due to the different potentials at the two ends of the electrode. The end which is connected to the other cathode has the same potential as the CB of that electrode, while the end which is in contact with the water has the potential intrinsic to SiC. This is to keep the fermi energy constant over the interfaces). The holes travel from the anode to the water, oxidicing the water: 

\[2H_2O + 4h^+ \rightarrow 4H^+ + O_2.\]

The electrons from the ehp travel from the anode to the cathode, where they interact with the hydrogen ions

\[4H^+ + 4e^- \rightarrow 2H_2\]

and the hydrogen gas can be collected. The process of creating oxygen and hydrogen gas from water requires an addition of energy corresponding to 1.23 eV per hydrogen molecule created. This is due to the change in Gibbs free energy before and after the reaction. This means that the minimum band gap of the electrode where the ehp are created is 1.23 eV. 

Another requirement of the system is that the semiconductor band energies are compatible with the electrochemical potentials of the water reduction and oxidation. In the NHE (normal hydrogen electrode) reference system, the reduction potential is 0 V. (This corresponds to the energy -4.5 eV relative the vacuum energy of the semiconductor band diagram. Cite Bard, Faulkner: Electrochemical methods - fundamentals and applications) Correspondingly the oxidation energy is 

\[-4.5 - 1.23 = -5.63 \,\, \mathrm{eV}\]

Both these energies must lie within the band gap of the electrode semiconductor. 

There are two problems that occur in these types of electrochemical cells. The first one is photocorrosion of the electrode, see above. The second one is recombination of O$_2$ and H$_2$ into water, a process called surface back reaction. There are other properties which affect the photocatalytic generation too: 

\begin{itemize}
\item Overpotentials
\item Charge separation
\item Mobility of carriers
\item Life time of carriers
\item Reflectivity of the electrode. 
\end{itemize}

\newpage
\section{VR-application}
\emph{Jianwu Sun}

The device used for water splitting is called an electrochemical cell. The idea is to use hydrogen as an energy source. This is promising because hydrogen has a very high energy density, even higher than that of gasoline (five times higher). 

Some important parameters to explore are: Carrier lifetime and transport properties, doping concentrations and their influences, Relation between pH and photocorrosion, best candidates for the cathode metal. Also the photocorrosion depending on doping needs to be investigated. The thought is that p-doped material will be better to resist corrosion. 

In ref. [8] they predict that the theoretical maximum efficiency f�r this type of structure is 16.8 \%. 

Some carrier equations that might come in useful: 

\[L_D = \sqrt{D\tau_R}\]
\[D = kT\mu/e\]

where L$_D$ is the diffusion lenght, $\tau_R$ is the carrier life time and D is the diffusion coefficient. 

\newpage
\section{Epitaxial p-type SiC as a self-driven photocathode for water splitting}
\emph{Kato, Masashi
Yasuda, Tomonari
Miyake, Keiko
Ichimura, Masaya
Hatayama, Tomoaki}

The currently highest reported efficiency of 3C SiC for water splitting is 0.38 \%. 

The Faraday efficiency is an important concept. It is the fraction of ehp:s that actually contribute to the extracted gas. Some carriers may be lost along the way, for example in creating other substances at the electrodes, or in surface back reaction where created hydrogen and oxygen gas recombine. 

The article states that dopant concentrations have been measured using C/V measurements, but it does not state how this has been done. \cite{Nobody06}




\bibliography{tst2-1}{}
\bibliographystyle{plain}


\end{document}








































