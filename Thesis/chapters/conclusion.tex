%-*- mode: LaTeX; -*-

\chapter{Conclusion}
Initially the idea was to investigate how C- and Si-face grown seeds differed in growth of B-doped samples. I was not able to reproducibly grow C-face seeds which were completely cubic. Investigating the seed growth I was able to show that however the temperature conditions were changed, 3C-SiC could not grow to cover the whole growth area on C-face substrates, while on the Si-face it was possible to reproducibly do so. The reason for this is not fully understood, but the reason may be the difference inherent in the surface of C-face and Si-face 4H-SiC, such as the surface energy of the surfaces. It may also be that the steps from the off-axis cut are different on C- and Si-faces, or contributed to the different form of the spirals. 

I have shown through optical microscopy that high B-doping leads to deteriorating surface quality of the sample. For most samples the surface quality worsened with increasing B-doping concentration of the source material. From optical microscopy and absorption measurements I have shown that it is possible to grow 3C-SiC material doped with boron using the sublimation growth technique described in this thesis. The absorption measurements indicate the boron to conduction band transition, but not the valence band to boron transition. The lack of the latter is thought to be due to the position of the Fermi level near or above the boron level in the band gap. 

I have shown using LTPL-spectroscopy that the boron levels in 3C-SiC grown in this work show no luminescence related to boron. The reason for this may be a combination of the deep level nature of the boron impurity with the poor crystal quality containing non-radiative defects. These PL-measurements further show inclusions of aluminium in the samples, which may form a competing relationship with the boron in carrier capture, further limiting the possibilities of luminescence. 