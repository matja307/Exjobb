%-*- mode: LaTeX; -*-

\chapter{Introduction}
%This thesis deals with SiC 
%Good properties when it comes to stability
%Used in power electronics, radiation (space - Citation needed!)

%SiC is a semiconductor
%Exists in a number of different polytypes
% This thesis deals with the cubic polytype
%One of the most common polytypes

% Difficulty in fabricating 3C with good quality
% Using sublimation epitaxy (FSGP)

% The band gap of 3C is good for solar cell applications
% And for water splitting
% Using doping to enhance the properties

% This thesis deals with examination of optical properties 
% List the characterization methods

% Describe the chapters

This thesis describes the growth and optical characterization of cubic silicon carbide (\emph{SiC}). SiC has attracted academic interest since the 19:th century, when it was first fabricated and used as an abrasive \cite{Acheson1893}. SiC has been found to be a very stable material. It exhibits a high chemical inertness \cite{Hume1941}, and is currently commonly used in high power and high temperature applications due to its ability to survive in such environments. 

SiC is a material which exists in a large number of different polytypes, the most common of which are hexagonal, cubic and rhombohedral. The work described in this thesis deals with the only cubic polytype, denoted \emph{3C}. This is one of the structurally most simple polytypes. Compared to the hexagonal counterparts 4H and 6H, the 3C polytype has for a long time been difficult to fabricate in good quality single crystal form, and is therefore less studied than the hexagonal types [Citation needed]. 