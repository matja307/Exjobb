%-*- mode: LaTeX; -*-

\chapter{Introduction}
%This thesis deals with SiC 
%Good properties when it comes to stability
%Used in power electronics, radiation (space - Citation needed!)

%SiC is a semiconductor
%Exists in a number of different polytypes
% This thesis deals with the cubic polytype
%One of the most common polytypes

% Difficulty in fabricating 3C with good quality
% Using sublimation epitaxy (FSGP)

% The band gap of 3C is good for solar cell applications
% And for water splitting
% Using doping to enhance the properties

% This thesis deals with examination of optical properties 
% List the characterization methods

% Describe the chapters

SiC is a semiconductor which has attracted interest in research and industry since the 19:th century, when it was first fabricated and used as an abrasive \cite{Acheson1893}. SiC has been found to be a very stable material. It exhibits a high chemical inertness \cite{Hume1941}, and is currently commonly used in high power, temperature and radiation applications due to its ability to survive in such environments \cite{J.B.CASADYandR.W.JOHNSON1996}. 

SiC is a material which exists in a large number of different polytypes, the most common of which are hexagonal, cubic and rhombohedral. The work described in this thesis deals with the only cubic polytype, denoted \emph{3C}. This is one of the structurally most simple polytypes. Compared to the hexagonal counterparts 4H- and 6H-SiC, the 3C-SiC polytype has for a long time been difficult to fabricate in good quality and large volume, and is therefore less studied than the hexagonal types. Recently a method of fabricating good quality free standing material using sublimation growth has been reported \cite{Jokubavicius2014}. The method has been used to grow the samples used in the work described in this thesis. With this method it is possible to grow free standing 3C-SiC, which means that the material is thick enough to exist as a free standing layer after mechanically polishing away the 4H-SiC substrate. 

Cubic SiC has many interesting material properties. It has a higher electron mobility compared to the common hexagonal polytypes. It has also attracted attention as a transistor material, due to low number of interface defects. One application for which 3C-SiC is well suited is the use of boron doped 3C-SiC in an intermediate band photovoltaic solar cell. This is suitable for 3C-SiC due to its band gap size together with the binding energy of boron as an acceptor in the material, which is almost ideal for photovoltaic cell material. This would give a significant increase in photovoltaic cell efficiency compared to the conventional single junction alternative \cite{Richards2003}[Current efficiencies citation needed]. Another proposed application of 3C-SiC is as a photo-electrode in a photoelectrochemical cell used for water splitting \cite{Kato2014,Yasuda2012}, where solar energy is used in the decomposition of water into hydrogen and oxygen gas. 

This thesis describes the growth and optical characterization of free standing cubic silicon carbide (\emph{SiC}).  The focus is on characterization of the optical properties of 3C-SiC. This is done using absorption spectroscopy, photoluminescence spectroscopy, \dots. Chapter \ref{sec:sic} gives an introduction to silicon carbide, its structure and properties. Chapter \ref{sec:growth} describes the process of growing the material. Both growth of undoped and boron doped material is described here. In chapter \ref{sec:characterization} a description of the different characterization methods is given, together with a theoretical description of what the measurements can tell about material properties. Chapter \ref{sec:experimental} describes how the experiments were done and chapter \ref{sec:results} describes the results obtained from the experiments. The results are discussed in chapter \ref{sec:discussion}. Chapters \ref{sec:conclusion} and \ref{sec:future} discuss what has been learned about the material and how the work should be continued in the future. 



































