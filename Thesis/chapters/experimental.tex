%-*- mode: LaTeX; -*-

\chapter{Experimental setup}
\label{sec:experimental_details}
In this chapter, the experimental setup details are explained. The chapter is divided into two parts. The first part describes how the samples were grown and prepared. The second part describes details on the experimental setups for the characterization experiments. 

\section{Growth and sample preparation}
\label{sec:experimental:samples}
All samples used in this work are grown using sublimation growth, as described in section \ref{sec:growth:fsgp}. The undoped samples are grown using two layers of XXX mm thick polycrystalline SiC source material. 4H-SiC substrates with 4$^\circ$ off axis surface were used for undoped growth. The substrates were chemically cleaned before growth. Typical size of the substrates were approximately 10 x 10 mm, and the thickness was XXX mm. Both the carbon and the silicon face of the 4H-SiC substrates were used during growth of undoped samples. 

The doped samples were grown using either direct or indirect doping methods, as described in section \ref{sec:growth:fsgp:doping}. The source material for doped growth was doped polycrystalline SiC, doped with boron concentrations of $10^{18}$, $10^{19}$ or $10^{20}$ cm$^{-3}$. This was used together with a piece of undoped source material. The doped samples were grown either homoepitaxially on an undoped seed, or on the silicon face of a 4H-SiC substrate. 

A piece of tantalum foil was placed in the bottom of the crucible for use as a carbon getter during growth of all samples. The source and substrate were separated by a 1 mm thick graphite spacer for all samples. The substrate was held in place on the spacer by a graphite plate placed on top. All samples were grown at pressures in the order of $10^{-4}$ to $10^{-5}$ mbar, varying during the growth process. A vacuum pump was connected to the reactor and running continuously during growth. The temperature and growth time were varied between samples, and are given for individual samples in section \ref{sec:results}. The temperature was changed by varying the power supplied by the RF generator to the copper coil. The reactor setup is described further in section \ref{sec:growth:fsgp}. Care was taken to place the insulating foam containing the crucible in the same position in the reactor for each run. 

After growth the reactor was cooled in vacuum, by instantly setting the RF-generator to 0 W when the growth time was over. Cooling time was reduced by the use of a fan placed outside of the reactor. Optical microscopy images were done on samples as grown, but any other characterization was done after chemically cleaning the samples. For the water splitting measurements contacts were placed on the samples. The contacts were done by XXX. 


\section{Characterization experiments}
\label{sec:experimental:characterization}
The absorption measurements were done with a PerkinElmer Lambda 950 UV/VIS setup. The software used was PerkinElmer UV WINLAB for molecular spectroscopy. Measurements were done in the range between 2000 to 380 nm. Samples were mounted on a black cardboard piece with a hole for light to pass through. Two hole sizes were used: XXX and YYY mm in diameter. Most samples were measured after mechanical polishing of the substrate, i.e. free-standing, but some samples were measured with the substrate remaining. The wavelength resolution of the spectrometer was 5 nm. 

Low temperature photoluminescence measurements were performed using a liquid helium cryostat. The helium was supercooled to 2.6 K using a pump. The wavelength resolution of the setup was approximately 0.5 Å. A laser of wavelength XXX was used. 








































