The properties of free-standing cubic silicon carbide for optoelectronic applications are explored in this work. The main focus of the work is on boron doped cubic silicon carbide, which is proposed as a highly useful material in several optoelectronic applications. The material is grown using sublimation epitaxy and the doped material is grown homoepitaxially on nominally undoped seeds. It is characterized using the experimental setups of photoluminescence spectroscopy, Nomarski interference spectroscopy and absorption spectroscopy. 

I study seed growth of nominally undoped cubic material on hexagonal (4H) substrates, and the influence on the grown material from the different faces of the substrate. It is found that it is not possible under the explored conditions to completely cover the growth area with the cubic polytype on the carbon face, but it can be done reproducibly on the silicon face. Reasons for this are discussed. Different doping setups are also explored.

The influence on the material properties from growth conditions is explored. It is shown from absorption measurements that it is possible to grow boron doped cubic silicon carbide using this growth method, but optical microscopy studies show that the sample quality degrades with high doping concentrations. 

I explore the luminescence properties of the material. No boron related emission is found with either room temperature or low temperature photoluminescence spectroscopy. Reasons for this are discussed using results from absorption measurements and optical microscopy. 

