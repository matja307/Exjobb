%-*- mode: LaTeX; -*-

\chapter{Future work}
To be able to investigate the growth-face influence on the B-doped material, further study of 3C-SiC growth on the C-face should be done. The pressure is a parameter which has not been explored in this regard. It is possible that by changing the growth pressure the results will improve. Studies of the influence of the surface on growth mechanisms will also be an important tool in understanding cubic growth on the C-face. Experiments such as electron microscopy will be of interest here. 

To be able to realize the proposed optoelectronical applications for B-doped 3C-SiC, it is necessary to understand the luminescence properties of the material. Studies of the defects in the grown material should be done, and ways to eliminate the them should be explored. This will include both finding better growth conditions and researching ways of improving the quality post-growth. Further studies of the relationship between the Al and B acceptor levels may give some more insight into the optical properties of the material. 

More applied studies of the material should also be undertaken in the future. To investigate the material as an electrode material in water splitting applications, photoelectrochemical measurements should be done. Use of boron doped 3C-SiC in photovoltaic applications will need studies of photocurrent generation. 

